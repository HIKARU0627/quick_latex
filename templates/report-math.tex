% 数学レポート用テンプレート
\documentclass[11pt,a4paper]{ltjsarticle}

% パッケージの読み込み
\usepackage{luatexja}
\usepackage{luatexja-fontspec}
\usepackage[margin=25mm]{geometry}
\usepackage{amsmath,amssymb,amsthm}
\usepackage{mathtools}
\usepackage{braket}
\usepackage{enumerate}
\usepackage{fancyhdr}
\usepackage{graphicx}
\usepackage{tikz}
\usepackage{pgfplots}
\usepackage{tcolorbox}
\usepackage{xcolor}
\usepackage{url}
\usepackage{hyperref}

% フォント設定
\setmainfont{Noto Serif CJK JP}
\setsansfont{Noto Sans CJK JP}
\setmonofont{Noto Sans Mono CJK JP}

% 数学環境の設定
\theoremstyle{definition}
\newtheorem{definition}{定義}[section]
\newtheorem{theorem}{定理}[section]
\newtheorem{lemma}[theorem]{補題}
\newtheorem{corollary}[theorem]{系}
\newtheorem{proposition}[theorem]{命題}

\theoremstyle{remark}
\newtheorem{remark}{注意}[section]
\newtheorem{example}{例}[section]

\theoremstyle{plain}
\newtheorem{problem}{問題}[section]
\newtheorem{exercise}{演習}[section]

% 証明環境の日本語化
\renewcommand{\proofname}{証明}

% 色の定義
\definecolor{accentblue}{RGB}{41,128,185}
\definecolor{accentgreen}{RGB}{39,174,96}
\definecolor{accentorange}{RGB}{230,126,34}
\definecolor{accentred}{RGB}{231,76,60}

% tcolorboxのスタイル定義
\newtcolorbox{mathframe}[1][]{
    colback=accentblue!10,
    colframe=accentblue,
    fonttitle=\bfseries,
    title={重要な数学概念},
    #1
}

\newtcolorbox{proofframe}[1][]{
    colback=accentgreen!10,
    colframe=accentgreen,
    fonttitle=\bfseries,
    title={証明のポイント},
    #1
}

\newtcolorbox{notebox}[1][]{
    colback=accentorange!10,
    colframe=accentorange,
    fonttitle=\bfseries,
    title={注意},
    #1
}

% 数学記号のショートカット
\newcommand{\N}{\mathbb{N}}
\newcommand{\Z}{\mathbb{Z}}
\newcommand{\Q}{\mathbb{Q}}
\newcommand{\R}{\mathbb{R}}
\newcommand{\C}{\mathbb{C}}
\newcommand{\F}{\mathbb{F}}

% ベクトル記法
\renewcommand{\vec}[1]{\boldsymbol{#1}}
\newcommand{\norm}[1]{\left\| #1 \right\|}
\newcommand{\abs}[1]{\left| #1 \right|}

% 微分・積分記法
\newcommand{\dx}{\,dx}
\newcommand{\dy}{\,dy}
\newcommand{\dz}{\,dz}
\newcommand{\dt}{\,dt}
\newcommand{\du}{\,du}
\newcommand{\dv}{\,dv}
\newcommand{\diff}[2]{\frac{d#1}{d#2}}
\newcommand{\pdiff}[2]{\frac{\partial #1}{\partial #2}}

% ヘッダー・フッター設定
\pagestyle{fancy}
\fancyhf{}
\fancyhead[L]{数学レポート}
\fancyhead[R]{\today}
\fancyfoot[C]{\thepage}

% ハイパーリンク設定
\hypersetup{
    colorlinks=true,
    linkcolor=accentblue,
    urlcolor=accentblue,
    citecolor=accentblue
}

% タイトル情報
\title{\Huge\textbf{数学レポート題目}}
\author{学籍番号:XXXXXXXX \\氏名:山田 太郎}
\date{\today}

\begin{document}

\maketitle

\tableofcontents
\clearpage

\section{はじめに}

本レポートでは、〇〇に関する数学的理論について考察する。特に、△△の性質と□□との関連性について詳細に分析し、その数学的意義を明らかにする。

\begin{mathframe}
このレポートで扱う主要な数学概念:
\begin{itemize}
    \item 概念1:具体的な定義や性質
    \item 概念2:関連する定理や応用
    \item 概念3:実際の計算例や証明
\end{itemize}
\end{mathframe}

\section{理論的背景}

\subsection{基本定義}

\begin{definition}[重要な概念]
実数体$\R$上のベクトル空間$V$において、内積を次のように定義する:
\begin{equation}
    \langle \vec{u}, \vec{v} \rangle : V \times V \to \R
\end{equation}
ここで、以下の性質を満たす:
\begin{enumerate}[(1)]
    \item 双線形性:$\langle a\vec{u}_1 + b\vec{u}_2, \vec{v} \rangle = a\langle \vec{u}_1, \vec{v} \rangle + b\langle \vec{u}_2, \vec{v} \rangle$
    \item 対称性:$\langle \vec{u}, \vec{v} \rangle = \langle \vec{v}, \vec{u} \rangle$
    \item 正定値性:$\langle \vec{v}, \vec{v} \rangle \geq 0$、等号は$\vec{v} = \vec{0}$のときのみ
\end{enumerate}
\end{definition}

\subsection{主要定理}

\begin{theorem}[コーシー・シュワルツの不等式]
\label{thm:cauchy-schwarz}
内積空間$(V, \langle \cdot, \cdot \rangle)$において、任意のベクトル$\vec{u}, \vec{v} \in V$に対して
\begin{equation}
    |\langle \vec{u}, \vec{v} \rangle|^2 \leq \langle \vec{u}, \vec{u} \rangle \langle \vec{v}, \vec{v} \rangle
\end{equation}
が成り立つ。等号は$\vec{u}$と$\vec{v}$が線形従属のときのみ成立する。
\end{theorem}

\begin{proof}
$\vec{v} = \vec{0}$の場合は自明である。$\vec{v} \neq \vec{0}$とする。

任意の実数$t$に対して、内積の正定値性より
\begin{equation}
    0 \leq \langle \vec{u} - t\vec{v}, \vec{u} - t\vec{v} \rangle
\end{equation}

\begin{proofframe}
証明の鍵となるアイデア:
\begin{itemize}
    \item 適切なパラメータ$t$を選んで二次式を構成
    \item 判別式を用いて不等式を導出
    \item 等号条件を線形従属性と関連付け
\end{itemize}
\end{proofframe}

双線形性を用いて展開すると:
\begin{align}
    0 &\leq \langle \vec{u}, \vec{u} \rangle - 2t\langle \vec{u}, \vec{v} \rangle + t^2\langle \vec{v}, \vec{v} \rangle
\end{align}

この二次式が任意の$t$で非負であるための条件は、判別式$D \leq 0$である:
\begin{equation}
    D = 4\langle \vec{u}, \vec{v} \rangle^2 - 4\langle \vec{u}, \vec{u} \rangle\langle \vec{v}, \vec{v} \rangle \leq 0
\end{equation}

よって求める不等式が得られる。\qed
\end{proof}

\section{具体的計算例}

\subsection{例1:ユークリッド空間での計算}

\begin{example}
$\R^3$において、ベクトル$\vec{u} = (1, 2, 3)$、$\vec{v} = (4, 5, 6)$に対してコーシー・シュワルツの不等式を確認する。

標準内積は:
\begin{equation}
    \langle \vec{u}, \vec{v} \rangle = 1 \cdot 4 + 2 \cdot 5 + 3 \cdot 6 = 4 + 10 + 18 = 32
\end{equation}

各ベクトルのノルムの二乗は:
\begin{align}
    \norm{\vec{u}}^2 &= 1^2 + 2^2 + 3^2 = 14 \\
    \norm{\vec{v}}^2 &= 4^2 + 5^2 + 6^2 = 77
\end{align}

コーシー・シュワルツの不等式を確認:
\begin{equation}
    32^2 = 1024 \leq 14 \times 77 = 1078 \quad \checkmark
\end{equation}
\end{example}

\subsection{例2:関数空間での応用}

連続関数の空間$C[0,1]$において、内積を
\begin{equation}
    \langle f, g \rangle = \int_0^1 f(x)g(x) \dx
\end{equation}
で定義する。

\begin{example}
$f(x) = x$、$g(x) = x^2$に対して:
\begin{align}
    \langle f, g \rangle &= \int_0^1 x \cdot x^2 \dx = \int_0^1 x^3 \dx = \left[ \frac{x^4}{4} \right]_0^1 = \frac{1}{4} \\
    \norm{f}^2 &= \int_0^1 x^2 \dx = \frac{1}{3} \\
    \norm{g}^2 &= \int_0^1 x^4 \dx = \frac{1}{5}
\end{align}

コーシー・シュワルツの不等式:
\begin{equation}
    \left(\frac{1}{4}\right)^2 = \frac{1}{16} \leq \frac{1}{3} \times \frac{1}{5} = \frac{1}{15} \quad \checkmark
\end{equation}
\end{example}

\section{発展的考察}

\subsection{一般化と応用}

\begin{notebox}
コーシー・シュワルツの不等式は以下のような様々な分野で応用される:
\begin{itemize}
    \item 確率論:確率変数の共分散に関する不等式
    \item 調和解析:フーリエ変換の性質
    \item 最適化:制約付き最適化問題
    \item 幾何学:角度と距離の関係
\end{itemize}
\end{notebox}

\subsection{数値計算による検証}

図\ref{fig:inequality-plot}に、様々なベクトルペアに対するコーシー・シュワルツの不等式の成立状況を示す。

\begin{figure}[htbp]
    \centering
    \begin{tikzpicture}
        \begin{axis}[
            xlabel={$\langle \vec{u}, \vec{v} \rangle^2$},
            ylabel={$\norm{\vec{u}}^2 \norm{\vec{v}}^2$},
            grid=major,
            legend pos=north west,
            width=10cm,
            height=8cm
        ]
        \addplot[blue, mark=*] coordinates {
            (1, 2) (4, 8) (9, 15) (16, 25) (25, 40) (36, 60) (49, 85) (64, 115)
        };
        \addplot[red, dashed] {x};
        \legend{データ点, $y = x$(等号線)}
        \end{axis}
    \end{tikzpicture}
    \caption{コーシー・シュワルツの不等式の視覚的確認}
    \label{fig:inequality-plot}
\end{figure}

\section{演習問題}

\begin{exercise}
次の問いに答えよ:
\begin{enumerate}[(1)]
    \item $\R^n$において、ベクトル$\vec{u} = (u_1, u_2, \ldots, u_n)$、$\vec{v} = (v_1, v_2, \ldots, v_n)$に対するコーシー・シュワルツの不等式を具体的に書け。
    \item 等号が成立する条件を線形代数の観点から説明せよ。
    \item 三角不等式$\norm{\vec{u} + \vec{v}} \leq \norm{\vec{u}} + \norm{\vec{v}}$をコーシー・シュワルツの不等式を用いて証明せよ。
\end{enumerate}
\end{exercise}

\begin{problem}
関数空間$L^2[0, 2\pi]$において、内積を
\begin{equation}
    \langle f, g \rangle = \int_0^{2\pi} f(x)g(x) \dx
\end{equation}
で定義する。$f(x) = \sin x$、$g(x) = \cos x$に対してコーシー・シュワルツの不等式を確認し、等号が成立しない理由を説明せよ。
\end{problem}

\section{まとめ}

本レポートでは、内積空間におけるコーシー・シュワルツの不等式について詳細に考察した。主な成果は以下の通りである:

\begin{enumerate}
    \item 定理の厳密な証明を判別式を用いて行った
    \item ユークリッド空間と関数空間における具体例を計算した
    \item 不等式の幾何学的・解析学的意味を明らかにした
    \item 数値計算による検証を通じて理論の妥当性を確認した
\end{enumerate}

\begin{mathframe}[title={今後の課題}]
\begin{itemize}
    \item より一般的な内積空間(複素数体上)での拡張
    \item ヘルダーの不等式やミンコフスキーの不等式との関連
    \item 関数解析への応用(ヒルベルト空間論)
\end{itemize}
\end{mathframe}

この不等式は現代数学の基礎として、線形代数、関数解析、確率論など幅広い分野で重要な役割を果たしている。

\begin{thebibliography}{99}
\bibitem{axler} S. Axler, \textit{Linear Algebra Done Right}, 3rd Edition, Springer, 2015.
\bibitem{rudin} W. Rudin, \textit{Principles of Mathematical Analysis}, 3rd Edition, McGraw-Hill, 1976.
\bibitem{halmos} P. R. Halmos, \textit{Finite-Dimensional Vector Spaces}, Springer, 1987.
\bibitem{japanese1} 齋藤正彦, 『線型代数入門』, 東京大学出版会, 1966.
\bibitem{japanese2} 杉浦光夫, 『解析入門I』, 東京大学出版会, 1980.
\end{thebibliography}

\end{document}