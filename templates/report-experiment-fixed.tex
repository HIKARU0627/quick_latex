\documentclass[12pt,a4paper]{ltjsarticle}

% ===== 日本語フォント設定(LuaLaTeX用) =====
\usepackage{luatexja-fontspec}

% ===== 基本パッケージ =====
\usepackage{amsmath,amssymb}
\usepackage{graphicx}
\usepackage{booktabs}
\usepackage{siunitx}
\usepackage{subcaption}
\usepackage{hyperref}
\usepackage{float}
\usepackage{pgfplots}
\usepackage{array}
\usepackage{multirow}
\usepackage{longtable}
\usepackage{adjustbox}

\pgfplotsset{compat=1.16}

% ===== SI単位の設定 =====
\sisetup{
    output-decimal-marker = {.},
    group-separator = {,},
    per-mode = symbol
}

% ===== 測定値用のカスタムコマンド =====
\newcommand{\measured}[3]{\SI{#1 \pm #2}{#3}}
\newcommand{\theoretical}[2]{\SI{#1}{#2}}

% ===== 測定データ用のテーブルスタイル =====
\newcolumntype{S}{S[table-format=3.2]}

% ===== ページ設定 =====
\setlength{\textwidth}{16cm}
\setlength{\textheight}{23cm}
\setlength{\oddsidemargin}{0cm}
\setlength{\evensidemargin}{0cm}
\setlength{\topmargin}{-0.5cm}

% ===== ハイパーリンク設定 =====
\hypersetup{
    colorlinks=true,
    linkcolor=blue,
    filecolor=magenta,
    urlcolor=cyan,
    citecolor=blue
}

% ===== 文書情報 =====
\title{【実験名】実験レポート}
\author{
    学籍番号: 12345678\\
    氏名: 山田 太郎\\
    共同実験者: 鈴木 花子\\
    実験日: 2024年X月X日\\
    提出日: \today
}
\date{}

% ===== 実験固有情報 =====
\newcommand{\labpartner}{鈴木 花子}
\newcommand{\experimentdate}{2024年X月X日}
\newcommand{\experimentlocation}{○○実験室}

\begin{document}

\maketitle

% ===== 実験情報 =====
\begin{center}
\begin{tabular}{ll}
共同実験者: & \labpartner \\
実験実施日: & \experimentdate \\
実験場所: & \experimentlocation \\
\end{tabular}
\end{center}
\vspace{1cm}

% ===== 概要 =====
\begin{abstract}
本実験では〇〇の測定を行い、理論値との比較を行った。
実験の結果、測定値は$\measured{9.81}{0.05}{\meter\per\second\squared}$となり、
理論値$\theoretical{9.80665}{\meter\per\second\squared}$とよく一致した。
誤差の要因として〇〇が考えられる。
\end{abstract}

% ===== 本文 =====
\section{目的}

本実験の目的は以下の通りである:
\begin{enumerate}
    \item 〇〇を測定し、その特性を理解する
    \item 理論値と実験値を比較し、誤差要因を考察する
    \item 測定技術を習得する
\end{enumerate}

\section{理論}

\subsection{基本原理}

〇〇の原理は以下の式で表される:

\begin{equation}
    F = ma = mg - kv
    \label{eq:basic}
\end{equation}

ここで、$m$は質量[$\si{\kilogram}$]、$a$は加速度[$\si{\meter\per\second\squared}$]、
$g$は重力加速度[$\si{\meter\per\second\squared}$]、$k$は抵抗係数[$\si{\kilogram\per\second}$]、
$v$は速度[$\si{\meter\per\second}$]である。

\subsection{理論値の導出}

式(\ref{eq:basic})より、終端速度$v_t$は:

\begin{equation}
    v_t = \frac{mg}{k}
    \label{eq:terminal}
\end{equation}

\section{実験方法}

\subsection{実験装置}

使用した装置を表\ref{tab:equipment}に示す。

\begin{table}[H]
    \centering
    \caption{使用機器一覧}
    \label{tab:equipment}
    \begin{tabular}{lll}
        \toprule
        機器名 & 型番 & 精度 \\
        \midrule
        電子天秤 & XYZ-123 & $\pm\SI{0.01}{\gram}$ \\
        ストップウォッチ & ABC-456 & $\pm\SI{0.01}{\second}$ \\
        定規 & --- & $\pm\SI{1}{\milli\meter}$ \\
        \bottomrule
    \end{tabular}
\end{table}

\subsection{実験手順}

\begin{enumerate}
    \item 装置を図\ref{fig:setup}のように組み立てる
    \item 質量$m$を電子天秤で測定する
    \item 高さ$h$から物体を落下させる
    \item 落下時間$t$をストップウォッチで測定する
    \item 手順3-4を10回繰り返す
\end{enumerate}

% 図を挿入する場合
% \begin{figure}[H]
%     \centering
%     \includegraphics[width=0.6\textwidth]{figures/setup.png}
%     \caption{実験装置の概略図}
%     \label{fig:setup}
% \end{figure}

\section{実験結果}

\subsection{測定データ}

測定結果を表\ref{tab:results}に示す。

\begin{table}[H]
    \centering
    \caption{落下時間の測定結果}
    \label{tab:results}
    \begin{tabular}{cS[table-format=1.2]S[table-format=1.2]}
        \toprule
        測定回数 & {時間 [$\si{\second}$]} & {速度 [$\si{\meter\per\second}$]} \\
        \midrule
        1  & 2.34 & 4.27 \\
        2  & 2.31 & 4.33 \\
        3  & 2.35 & 4.26 \\
        4  & 2.33 & 4.29 \\
        5  & 2.32 & 4.31 \\
        6  & 2.34 & 4.27 \\
        7  & 2.33 & 4.29 \\
        8  & 2.35 & 4.26 \\
        9  & 2.31 & 4.33 \\
        10 & 2.32 & 4.31 \\
        \midrule
        平均 & 2.33 & 4.29 \\
        標準偏差 & 0.015 & 0.028 \\
        \bottomrule
    \end{tabular}
\end{table}

\subsection{データ解析}

平均値と標準誤差:
\begin{equation}
    \bar{t} = \SI{2.33 \pm 0.005}{\second}
\end{equation}

ここで、標準誤差は$\sigma/\sqrt{n} = 0.015/\sqrt{10} = 0.005$である。

\subsection{グラフ}

% PGFPlotsを使用したグラフの例
\begin{figure}[H]
    \centering
    \begin{tikzpicture}
        \begin{axis}[
            xlabel={時間 [$\si{\second}$]},
            ylabel={距離 [$\si{\meter}$]},
            grid=major,
            width=10cm,
            height=7cm,
        ]
        \addplot[
            color=blue,
            mark=o,
            only marks,
            error bars/.cd,
            y dir=both,
            y explicit,
        ] coordinates {
            (0,0) +- (0,0.1)
            (1,4.9) +- (0,0.2)
            (2,19.6) +- (0,0.3)
            (3,44.1) +- (0,0.4)
        };
        \addplot[
            color=red,
            domain=0:3,
        ] {4.9*x^2};
        \legend{実験値, 理論値}
        \end{axis}
    \end{tikzpicture}
    \caption{落下距離と時間の関係}
    \label{fig:graph}
\end{figure}

\section{考察}

\subsection{誤差の評価}

理論値との相対誤差は:
\begin{equation}
    \text{相対誤差} = \frac{|実験値 - 理論値|}{理論値} \times 100 = 0.3\%
\end{equation}

この誤差は十分小さく、実験は成功したと言える。

\subsection{誤差要因}

考えられる誤差要因:
\begin{enumerate}
    \item 空気抵抗の影響
    \item 測定者の反応時間(約$\SI{0.2}{\second}$)
    \item 装置の振動
\end{enumerate}

\subsection{改善点}

\begin{itemize}
    \item 光電センサーを用いた自動測定
    \item 真空中での実験
    \item 測定回数の増加
\end{itemize}

\section{結論}

本実験により、〇〇を測定し、理論値とよく一致する結果を得た。
測定値は$\SI{9.81 \pm 0.05}{\meter\per\second\squared}$であり、
理論値$\SI{9.80665}{\meter\per\second\squared}$との相対誤差は0.3\%であった。

今後の課題として、より精密な測定装置を用いることで、
さらに高精度な測定が可能になると考えられる。

\section{参考文献}

\begin{thebibliography}{9}
\bibitem{ref1} 物理学実験指導書, ○○大学物理学科, 2024.
\bibitem{ref2} 力学の教科書, 著者名, 出版社, 2023.
\end{thebibliography}

\end{document}