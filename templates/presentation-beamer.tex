\documentclass[aspectratio=169,12pt]{beamer}

% ===== 日本語フォント設定(LuaLaTeX用) =====
\usepackage{luatexja-fontspec}
% \setmainjfont{Noto Sans CJK JP}  % 必要に応じて設定

% ===== パッケージ =====
\usepackage{amsmath,amssymb}
\usepackage{graphicx}
\usepackage{listings}
\usepackage{xcolor}
\usepackage{algorithm2e}
\usepackage{booktabs}
\usepackage{tikz}
\usepackage{pgfplots}
\pgfplotsset{compat=1.16}

% ===== テーマ設定 =====
\usetheme{Madrid}
\usecolortheme{default}

% ===== カスタム色定義 =====
\definecolor{primaryblue}{RGB}{25, 102, 179}
\definecolor{secondaryblue}{RGB}{52, 152, 219}
\definecolor{accentorange}{RGB}{243, 156, 18}
\definecolor{darkgray}{RGB}{52, 58, 64}

% ===== 色設定の適用 =====
\setbeamercolor{palette primary}{bg=primaryblue,fg=white}
\setbeamercolor{palette secondary}{bg=secondaryblue,fg=white}
\setbeamercolor{palette tertiary}{bg=darkgray,fg=white}
\setbeamercolor{palette quaternary}{bg=accentorange,fg=white}

\setbeamercolor{structure}{fg=primaryblue}
\setbeamercolor{frametitle}{bg=primaryblue,fg=white}
\setbeamercolor{title}{fg=primaryblue}

% ===== フォント設定 =====
\setbeamerfont{title}{size=\Large,series=\bfseries}
\setbeamerfont{frametitle}{size=\large,series=\bfseries}
\setbeamerfont{framesubtitle}{size=\small}

% ===== コード表示設定 =====
\definecolor{codegreen}{rgb}{0,0.6,0}
\definecolor{codegray}{rgb}{0.5,0.5,0.5}
\definecolor{codepurple}{rgb}{0.58,0,0.82}
\definecolor{backcolour}{rgb}{0.95,0.95,0.92}

\lstdefinestyle{mystyle}{
    backgroundcolor=\color{backcolour},   
    commentstyle=\color{codegreen},
    keywordstyle=\color{magenta},
    numberstyle=\tiny\color{codegray},
    stringstyle=\color{codepurple},
    basicstyle=\ttfamily\footnotesize,
    breakatwhitespace=false,         
    breaklines=true,                 
    keepspaces=true,                 
    numbers=left,                    
    numbersep=5pt,                  
    showspaces=false,                
    showstringspaces=false,
    showtabs=false,                  
    tabsize=2
}
\lstset{style=mystyle}

% ===== TikZ設定 =====
\usetikzlibrary{shapes,arrows,positioning,calc}

% ===== プレゼンテーション情報 =====
\title[短いタイトル]{プレゼンテーションタイトル}
\subtitle{サブタイトル}
\author[山田]{山田 太郎}
\institute[○○大学]{○○大学大学院 ○○研究科}
\date{\today}

% ===== カスタムコマンド =====
\newcommand{\highlight}[1]{\textcolor{accentorange}{\textbf{#1}}}
\newcommand{\important}[1]{\textcolor{red}{\textbf{#1}}}

\begin{document}

% ===== タイトルページ =====
\frame{\titlepage}

% ===== 目次 =====
\begin{frame}{目次}
    \tableofcontents
\end{frame}

% ===== セクション1:イントロダクション =====
\section{はじめに}

\begin{frame}{研究背景}
    \begin{itemize}
        \item<1-> \highlight{背景1}:〇〇の重要性が高まっている
        \item<2-> \highlight{背景2}:従来手法では△△が課題
        \item<3-> \highlight{背景3}:□□の需要が急増している
    \end{itemize}
    
    \vspace{1em}
    \onslide<4->{
    \begin{exampleblock}{具体例}
        例えば、〇〇分野では年間△△\%の成長率を記録している。
    \end{exampleblock}
    }
\end{frame}

\begin{frame}{研究課題}
    \begin{columns}
        \begin{column}{0.6\textwidth}
            \textbf{現在の問題点:}
            \begin{enumerate}
                \item 処理速度が遅い
                \item 精度が不十分
                \item スケーラビリティに欠ける
            \end{enumerate}
            
            \vspace{1em}
            
            \textbf{解決すべき課題:}
            \begin{itemize}
                \item より高速な処理の実現
                \item 精度の大幅な向上
                \item 大規模データへの対応
            \end{itemize}
        \end{column}
        \begin{column}{0.4\textwidth}
            % \begin{figure}
            %     \centering
            %     \includegraphics[width=\textwidth]{figures/problem_overview.png}
            %     \caption{課題の概要}
            % \end{figure}
            \textbf{図表:}課題の概要図\\
            (実際の図を挿入)
        \end{column}
    \end{columns}
\end{frame}

\begin{frame}{研究目的}
    \begin{alertblock}{本研究の目的}
        従来手法の問題を解決し、\important{高性能で実用的な〇〇システム}を開発する
    \end{alertblock}
    
    \vspace{1em}
    
    \textbf{具体的な目標:}
    \begin{enumerate}
        \item 処理速度を\highlight{10倍}向上させる
        \item 精度を\highlight{95\%以上}に改善する
        \item 100万件規模のデータに対応する
    \end{enumerate}
\end{frame}

% ===== セクション2:提案手法 =====
\section{提案手法}

\begin{frame}{提案手法の概要}
    \begin{center}
        \begin{tikzpicture}[node distance=2cm]
            \node[draw, rectangle, fill=primaryblue!20] (input) {入力データ};
            \node[draw, rectangle, fill=secondaryblue!20, right of=input] (preprocess) {前処理};
            \node[draw, rectangle, fill=accentorange!20, right of=preprocess] (algorithm) {メインアルゴリズム};
            \node[draw, rectangle, fill=darkgray!20, right of=algorithm] (output) {出力};
            
            \draw[->] (input) -- (preprocess);
            \draw[->] (preprocess) -- (algorithm);
            \draw[->] (algorithm) -- (output);
        \end{tikzpicture}
    \end{center}
    
    \vspace{1em}
    
    \textbf{主な特徴:}
    \begin{itemize}
        \item \highlight{革新的なアルゴリズム}の採用
        \item \highlight{効率的な前処理}による高速化
        \item \highlight{自動最適化機能}による精度向上
    \end{itemize}
\end{frame}

\begin{frame}{アルゴリズムの詳細}
    \begin{columns}
        \begin{column}{0.5\textwidth}
            \textbf{Step 1: データ前処理}
            \begin{equation}
                x' = \alpha x + \beta
            \end{equation}
            
            \textbf{Step 2: 特徴抽出}
            \begin{equation}
                f = \text{extract}(x')
            \end{equation}
            
            \textbf{Step 3: 分類/回帰}
            \begin{equation}
                y = \text{classify}(f)
            \end{equation}
        \end{column}
        \begin{column}{0.5\textwidth}
            \begin{figure}
                \centering
                \begin{tikzpicture}[scale=0.8]
                    \draw[->] (0,0) -- (4,0) node[right] {特徴1};
                    \draw[->] (0,0) -- (0,3) node[above] {特徴2};
                    \fill[blue] (1,1) circle (2pt);
                    \fill[blue] (1.5,1.2) circle (2pt);
                    \fill[red] (2.5,2) circle (2pt);
                    \fill[red] (3,2.2) circle (2pt);
                    \draw[thick] (0.5,0.5) -- (3.5,2.5);
                    \node[below] at (2,0) {決定境界};
                \end{tikzpicture}
                \caption{分類の概念図}
            \end{figure}
        \end{column}
    \end{columns}
\end{frame}

\begin{frame}[fragile]{実装例}
    \begin{lstlisting}[language=Python]
def proposed_algorithm(data):
    """提案アルゴリズムの実装"""
    # Step 1: 前処理
    processed_data = preprocess(data)
    
    # Step 2: 特徴抽出
    features = extract_features(processed_data)
    
    # Step 3: 予測
    predictions = classify(features)
    
    return predictions

def preprocess(data):
    """データの前処理"""
    return normalize(data)

def extract_features(data):
    """特徴抽出"""
    return apply_transformation(data)
    \end{lstlisting}
\end{frame}

% ===== セクション3:実験と評価 =====
\section{実験と評価}

\begin{frame}{実験設定}
    \begin{itemize}
        \item \textbf{データセット}:〇〇データセット(10,000サンプル)
        \item \textbf{評価指標}:精度、処理時間、メモリ使用量
        \item \textbf{比較手法}:従来手法A, B, C
        \item \textbf{実行環境}:Ubuntu 20.04, Python 3.8, 32GB RAM
    \end{itemize}
    
    \vspace{1em}
    
    \begin{exampleblock}{実験プロトコル}
        \begin{enumerate}
            \item データを8:2で学習用・テスト用に分割
            \item 5-fold交差検証を実施
            \item 各手法を10回実行して平均値を算出
        \end{enumerate}
    \end{exampleblock}
\end{frame}

\begin{frame}{精度比較結果}
    \begin{center}
        \begin{tikzpicture}
            \begin{axis}[
                ybar,
                xlabel={手法},
                ylabel={精度 (\%)},
                ymin=70,
                ymax=100,
                bar width=15pt,
                width=10cm,
                height=6cm,
                xtick=data,
                xticklabels={従来手法A, 従来手法B, 従来手法C, 提案手法},
                legend style={at={(0.5,-0.15)}, anchor=north},
                grid=major
            ]
            \addplot coordinates {(1,78.5) (2,82.3) (3,85.1) (4,92.7)};
            \end{axis}
        \end{tikzpicture}
    \end{center}
    
    \textbf{結果:}提案手法が\highlight{最高精度92.7\%}を達成!
\end{frame}

\begin{frame}{処理時間比較}
    \begin{columns}
        \begin{column}{0.5\textwidth}
            \begin{table}
                \centering
                \caption{処理時間比較}
                \begin{tabular}{lr}
                    \toprule
                    手法 & 時間 (秒) \\
                    \midrule
                    従来手法A & 15.2 \\
                    従来手法B & 12.8 \\
                    従来手法C & 8.9 \\
                    \textbf{提案手法} & \textbf{2.3} \\
                    \bottomrule
                \end{tabular}
            \end{table}
        \end{column}
        \begin{column}{0.5\textwidth}
            \begin{alertblock}{性能向上}
                \begin{itemize}
                    \item 従来手法Aの\highlight{6.6倍}高速
                    \item 従来手法Bの\highlight{5.6倍}高速
                    \item 従来手法Cの\highlight{3.9倍}高速
                \end{itemize}
            \end{alertblock}
        \end{column}
    \end{columns}
\end{frame}

\begin{frame}{スケーラビリティテスト}
    \begin{center}
        \begin{tikzpicture}
            \begin{axis}[
                xlabel={データサイズ},
                ylabel={処理時間 (秒)},
                width=10cm,
                height=6cm,
                legend style={at={(0.5,1.02)}, anchor=south},
                grid=major
            ]
            \addplot[blue,mark=o] coordinates {
                (1000,0.5) (5000,2.1) (10000,4.2) (50000,18.5) (100000,35.2)
            };
            \addplot[red,mark=square] coordinates {
                (1000,0.1) (5000,0.4) (10000,0.8) (50000,3.2) (100000,6.1)
            };
            \legend{従来手法,提案手法}
            \end{axis}
        \end{tikzpicture}
    \end{center}
    
    提案手法は大規模データでも\highlight{優れた性能}を維持
\end{frame}

% ===== セクション4:まとめ =====
\section{まとめ}

\begin{frame}{研究成果}
    \begin{itemize}
        \item<1-> \highlight{新しいアルゴリズム}を提案し、実装
        \item<2-> \highlight{精度を92.7\%}まで向上(従来比7.6ポイント改善)
        \item<3-> \highlight{処理速度を最大6.6倍}高速化
        \item<4-> \highlight{大規模データ}への対応を実現
    \end{itemize}
    
    \vspace{2em}
    
    \onslide<5->{
    \begin{exampleblock}{学術的貢献}
        \begin{itemize}
            \item 理論的新規性の証明
            \item 実用性の実証
            \item オープンソースでの公開予定
        \end{itemize}
    \end{exampleblock}
    }
\end{frame}

\begin{frame}{今後の課題}
    \textbf{短期的課題:}
    \begin{itemize}
        \item より多様なデータセットでの評価
        \item パラメータ調整の自動化
        \item ユーザビリティの改善
    \end{itemize}
    
    \vspace{1em}
    
    \textbf{長期的課題:}
    \begin{itemize}
        \item リアルタイム処理への対応
        \item 分散処理システムの構築
        \item 産業応用の検討
    \end{itemize}
    
    \vspace{1em}
    
    \begin{alertblock}{将来展望}
        本研究成果を基盤として、〇〇分野の発展に貢献したい
    \end{alertblock}
\end{frame}

\begin{frame}{質疑応答}
    \begin{center}
        {\Huge ご質問・ご討論}
        
        \vspace{2em}
        
        {\Large ありがとうございました}
        
        \vspace{2em}
        
        \textbf{連絡先:}\\
        yamada@example.university.ac.jp
        
        \vspace{1em}
        
        \textbf{コード公開:}\\
        \url{https://github.com/yamada/research}
    \end{center}
\end{frame}

% ===== 付録(必要に応じて) =====
\appendix

\begin{frame}{付録:詳細な実験結果}
    \begin{table}
        \centering
        \caption{詳細な性能比較}
        \footnotesize
        \begin{tabular}{lcccc}
            \toprule
            手法 & 精度 & 適合率 & 再現率 & F1スコア \\
            \midrule
            従来手法A & 78.5 & 76.2 & 80.1 & 78.1 \\
            従来手法B & 82.3 & 81.0 & 83.7 & 82.3 \\
            従来手法C & 85.1 & 84.5 & 85.8 & 85.1 \\
            提案手法 & \textbf{92.7} & \textbf{92.1} & \textbf{93.2} & \textbf{92.6} \\
            \bottomrule
        \end{tabular}
    \end{table}
\end{frame}

\begin{frame}{付録:アルゴリズムの詳細}
    \begin{algorithm}[H]
        \caption{提案アルゴリズム(詳細版)}
        \KwIn{データセット $D$, パラメータ $\theta$}
        \KwOut{予測結果 $Y$}
        \BlankLine
        $Y \leftarrow \emptyset$\;
        \For{$x \in D$}{
            $x' \leftarrow \text{preprocess}(x)$\;
            $f \leftarrow \text{extract\_features}(x')$\;
            $y \leftarrow \text{predict}(f, \theta)$\;
            $Y \leftarrow Y \cup \{y\}$\;
        }
        \Return{$Y$}\;
    \end{algorithm}
\end{frame}

\end{document}