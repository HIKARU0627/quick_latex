\documentclass[12pt,a4paper]{ltjsarticle}

% ===== 日本語フォント設定(LuaLaTeX用) =====
\usepackage{luatexja-fontspec}
% \setmainjfont{Noto Serif CJK JP}  % 必要に応じて設定

% ===== パッケージ =====
\usepackage{amsmath,amssymb}
\usepackage{graphicx}
\usepackage{hyperref}
\usepackage{listings}
\usepackage{xcolor}
\usepackage{fancyvrb}
\usepackage{algorithm2e}
\usepackage{booktabs}
\usepackage{float}
\usepackage{subcaption}
\usepackage{url}

% ===== ページ設定 =====
\setlength{\textwidth}{16cm}
\setlength{\textheight}{23cm}
\setlength{\oddsidemargin}{0cm}
\setlength{\evensidemargin}{0cm}
\setlength{\topmargin}{-0.5cm}

% ===== 色の定義 =====
\definecolor{codegreen}{rgb}{0,0.6,0}
\definecolor{codegray}{rgb}{0.5,0.5,0.5}
\definecolor{codepurple}{rgb}{0.58,0,0.82}
\definecolor{backcolour}{rgb}{0.95,0.95,0.92}

% ===== Python用コード設定 =====
\lstdefinestyle{python}{
    backgroundcolor=\color{backcolour},   
    commentstyle=\color{codegreen},
    keywordstyle=\color{magenta},
    numberstyle=\tiny\color{codegray},
    stringstyle=\color{codepurple},
    basicstyle=\ttfamily\footnotesize,
    breakatwhitespace=false,         
    breaklines=true,                 
    captionpos=b,                    
    keepspaces=true,                 
    numbers=left,                    
    numbersep=5pt,                  
    showspaces=false,                
    showstringspaces=false,
    showtabs=false,                  
    tabsize=4,
    language=Python
}

% ===== Java用コード設定 =====
\lstdefinestyle{java}{
    backgroundcolor=\color{backcolour},   
    commentstyle=\color{codegreen},
    keywordstyle=\color{blue},
    numberstyle=\tiny\color{codegray},
    stringstyle=\color{red},
    basicstyle=\ttfamily\footnotesize,
    breakatwhitespace=false,         
    breaklines=true,                 
    captionpos=b,                    
    keepspaces=true,                 
    numbers=left,                    
    numbersep=5pt,                  
    showspaces=false,                
    showstringspaces=false,
    showtabs=false,                  
    tabsize=4,
    language=Java
}

% ===== C/C++用コード設定 =====
\lstdefinestyle{cpp}{
    backgroundcolor=\color{backcolour},   
    commentstyle=\color{codegreen},
    keywordstyle=\color{blue},
    numberstyle=\tiny\color{codegray},
    stringstyle=\color{red},
    basicstyle=\ttfamily\footnotesize,
    breakatwhitespace=false,         
    breaklines=true,                 
    captionpos=b,                    
    keepspaces=true,                 
    numbers=left,                    
    numbersep=5pt,                  
    showspaces=false,                
    showstringspaces=false,
    showtabs=false,                  
    tabsize=4,
    language=C++
}

% ===== アルゴリズム設定 =====
\SetAlgoLined
\SetKwInOut{Input}{入力}
\SetKwInOut{Output}{出力}

% ===== 文書情報 =====
\title{【授業名】\\プログラミング課題レポート}
\author{
    学籍番号: 12345678\\
    氏名: 山田 太郎\\
    提出日: \today
}
\date{}

\begin{document}

\maketitle

% ===== 概要 =====
\begin{abstract}
本レポートでは、〇〇の実装について報告する。
課題要求を分析し、アルゴリズムを設計・実装した。
実装した〇〇アルゴリズムの時間計算量は$O(n)$であり、
テストケースすべてで期待通りの動作を確認した。
\end{abstract}

% ===== 本文 =====
\section{課題内容}

\subsection{問題設定}
与えられた課題は以下の通りである:

\begin{quote}
〇〇を実装せよ。入力として〇〇が与えられ、出力として〇〇を返すプログラムを作成すること。
\end{quote}

\subsection{要求仕様}
\begin{itemize}
    \item 入力:〇〇(例:整数配列、文字列など)
    \item 出力:〇〇(例:ソート済み配列、検索結果など)
    \item 制約:〇〇(例:配列サイズは1000以下)
    \item 評価基準:正確性、効率性、可読性
\end{itemize}

\section{アルゴリズム設計}

\subsection{基本方針}
問題を解決するため、以下のアプローチを採用した:

\begin{enumerate}
    \item 問題を〇〇に分解
    \item 〇〇アルゴリズムを適用
    \item 効率化のため〇〇を工夫
\end{enumerate}

\subsection{アルゴリズム}
実装したアルゴリズムを擬似コードで示す:

\begin{algorithm}[H]
\caption{〇〇アルゴリズム}
\Input{配列 $A[1..n]$}
\Output{処理結果}
\BlankLine
初期化処理\;
\For{$i \leftarrow 1$ \KwTo $n$}{
    \If{条件}{
        処理A\;
    }
    \Else{
        処理B\;
    }
}
結果を返す\;
\end{algorithm}

\subsection{計算量解析}
\begin{itemize}
    \item \textbf{時間計算量}:$O(n \log n)$
    \begin{itemize}
        \item 主ループ:$O(n)$
        \item 内部処理:$O(\log n)$
    \end{itemize}
    \item \textbf{空間計算量}:$O(n)$
    \begin{itemize}
        \item 配列格納:$O(n)$
        \item 作業領域:$O(1)$
    \end{itemize}
\end{itemize}

\section{実装}

\subsection{メインプログラム}
以下にPythonによる実装を示す:

\begin{lstlisting}[style=python, caption=メインプログラム (main.py)]
def solve_problem(input_data):
    """
    問題を解決するメイン関数
    
    Args:
        input_data: 入力データ
    
    Returns:
        解決結果
    """
    # データの前処理
    processed_data = preprocess(input_data)
    
    # メインアルゴリズムの実行
    result = main_algorithm(processed_data)
    
    # 後処理
    final_result = postprocess(result)
    
    return final_result

def preprocess(data):
    """データの前処理"""
    # 入力データの検証
    if not validate_input(data):
        raise ValueError("Invalid input data")
    
    # 正規化処理
    normalized = normalize_data(data)
    return normalized

def main_algorithm(data):
    """メインアルゴリズム"""
    result = []
    
    for item in data:
        # 処理ロジック
        processed_item = process_item(item)
        result.append(processed_item)
    
    return result

def process_item(item):
    """個別アイテムの処理"""
    # 具体的な処理内容
    return item * 2  # 例:2倍にする

if __name__ == "__main__":
    # テストデータ
    test_input = [1, 2, 3, 4, 5]
    result = solve_problem(test_input)
    print(f"Result: {result}")
\end{lstlisting}

\subsection{ヘルパー関数}
補助的な関数の実装:

\begin{lstlisting}[style=python, caption=ヘルパー関数 (utils.py)]
def validate_input(data):
    """
    入力データの妥当性検証
    """
    if not isinstance(data, list):
        return False
    if len(data) == 0:
        return False
    return all(isinstance(item, (int, float)) for item in data)

def normalize_data(data):
    """
    データの正規化
    """
    if not data:
        return []
    
    # 平均値による正規化
    mean_val = sum(data) / len(data)
    return [(x - mean_val) for x in data]

def format_output(result):
    """
    出力フォーマット
    """
    return f"処理結果: {result}"
\end{lstlisting}

\section{テスト}

\subsection{テストケース設計}
以下のテストケースを設計・実行した:

\begin{table}[H]
    \centering
    \caption{テストケース一覧}
    \begin{tabular}{clll}
        \toprule
        ID & 入力 & 期待出力 & 結果 \\
        \midrule
        TC1 & [1, 2, 3] & [2, 4, 6] & ✓ PASS \\
        TC2 & [] & [] & ✓ PASS \\
        TC3 & [0] & [0] & ✓ PASS \\
        TC4 & [−1, 0, 1] & [−2, 0, 2] & ✓ PASS \\
        TC5 & 大規模データ(n=1000) & 期待結果 & ✓ PASS \\
        \bottomrule
    \end{tabular}
    \label{tab:testcases}
\end{table}

\subsection{テストコード}
単体テストの実装:

\begin{lstlisting}[style=python, caption=テストコード (test.py)]
import unittest
from main import solve_problem, validate_input

class TestSolution(unittest.TestCase):
    
    def test_basic_case(self):
        """基本的なテストケース"""
        input_data = [1, 2, 3]
        expected = [2, 4, 6]
        result = solve_problem(input_data)
        self.assertEqual(result, expected)
    
    def test_empty_input(self):
        """空入力のテスト"""
        input_data = []
        expected = []
        result = solve_problem(input_data)
        self.assertEqual(result, expected)
    
    def test_single_element(self):
        """単一要素のテスト"""
        input_data = [5]
        expected = [10]
        result = solve_problem(input_data)
        self.assertEqual(result, expected)
    
    def test_validation(self):
        """入力検証のテスト"""
        self.assertTrue(validate_input([1, 2, 3]))
        self.assertFalse(validate_input("invalid"))
        self.assertFalse(validate_input(None))

if __name__ == '__main__':
    unittest.main()
\end{lstlisting}

\section{実行結果}

\subsection{出力例}
プログラムの実行結果を以下に示す:

\begin{Verbatim}[frame=single, fontsize=\small]
$ python main.py
入力: [1, 2, 3, 4, 5]
処理開始...
前処理完了
アルゴリズム実行中...
後処理完了
Result: [2, 4, 6, 8, 10]
実行時間: 0.001秒
\end{Verbatim}

\subsection{性能測定}
様々なデータサイズでの性能測定結果:

\begin{table}[H]
    \centering
    \caption{性能測定結果}
    \begin{tabular}{rrr}
        \toprule
        データサイズ & 実行時間 (秒) & メモリ使用量 (MB) \\
        \midrule
        100 & 0.001 & 1.2 \\
        1,000 & 0.010 & 2.5 \\
        10,000 & 0.098 & 12.3 \\
        100,000 & 0.987 & 85.4 \\
        \bottomrule
    \end{tabular}
    \label{tab:performance}
\end{table}

\section{考察}

\subsection{アルゴリズムの評価}
実装したアルゴリズムについて以下の点で評価できる:

\begin{itemize}
    \item \textbf{正確性}:全テストケースで期待通りの結果
    \item \textbf{効率性}:理論計算量$O(n \log n)$を達成
    \item \textbf{可読性}:関数分割により理解しやすい構造
\end{itemize}

\subsection{改善点}
今後の改善可能な点:

\begin{enumerate}
    \item エラーハンドリングの強化
    \item より大規模データへの対応
    \item 並列処理による高速化
\end{enumerate}

\subsection{学んだこと}
本課題を通じて学んだこと:

\begin{itemize}
    \item アルゴリズム設計の重要性
    \item テスト駆動開発の有効性
    \item 計算量解析の実用性
\end{itemize}

\section{結論}

本レポートでは〇〇の実装を行い、要求仕様を満たすプログラムを作成した。
実装したアルゴリズムは効率的で正確であり、すべてのテストケースをパスした。

今後は実際のアプリケーションでの適用を検討し、
さらなる最適化を進めていきたい。

% ===== 付録 =====
\section*{付録}

\subsection*{A. 完全なソースコード}
完全なソースコードは以下のリポジトリで公開している:
\url{https://github.com/username/project}

\subsection*{B. 実行環境}
\begin{itemize}
    \item OS: Ubuntu 22.04 LTS
    \item Python: 3.9.7
    \item メモリ: 16GB RAM
    \item CPU: Intel Core i7-9750H
\end{itemize}

% ===== 参考文献 =====
\begin{thebibliography}{9}
\bibitem{algo} T.H. Cormen et al., ``Introduction to Algorithms,'' MIT Press, 2009.
\bibitem{python} M. Lutz, ``Learning Python,'' O'Reilly Media, 2013.
\bibitem{testing} K. Beck, ``Test Driven Development,'' Addison-Wesley, 2003.
\end{thebibliography}

\end{document}