\documentclass[12pt,a4paper]{ltjsarticle}

% ===== 日本語フォント設定(LuaLaTeX用) =====
\usepackage{luatexja-fontspec}

% ===== 基本パッケージ =====
\usepackage{amsmath,amssymb,amsthm}
\usepackage{graphicx}
\usepackage{hyperref}
\usepackage{listings}
\usepackage{xcolor}
\usepackage{enumerate}
\usepackage{fancyvrb}
\usepackage{multicol}
\usepackage{booktabs}
\usepackage{float}

% ===== 色の定義 =====
\definecolor{codegreen}{rgb}{0,0.6,0}
\definecolor{codegray}{rgb}{0.5,0.5,0.5}
\definecolor{codepurple}{rgb}{0.58,0,0.82}
\definecolor{backcolour}{rgb}{0.95,0.95,0.92}

% ===== コード表示設定 =====
\lstdefinestyle{python}{
    language=Python,
    backgroundcolor=\color{backcolour},   
    commentstyle=\color{codegreen},
    keywordstyle=\color{magenta},
    numberstyle=\tiny\color{codegray},
    stringstyle=\color{codepurple},
    basicstyle=\ttfamily\footnotesize,
    breakatwhitespace=false,         
    breaklines=true,                 
    captionpos=b,                    
    keepspaces=true,                 
    numbers=left,                    
    numbersep=5pt,                  
    showspaces=false,                
    showstringspaces=false,
    showtabs=false,                  
    tabsize=4
}

\lstset{style=python}

% ===== 定理環境の定義 =====
\theoremstyle{definition}
\newtheorem{definition}{定義}[section]
\newtheorem{theorem}{定理}[section]
\newtheorem{lemma}{補題}[section]
\newtheorem{example}{例}[section]

% ===== ページ設定 =====
\setlength{\textwidth}{16cm}
\setlength{\textheight}{23cm}
\setlength{\oddsidemargin}{0cm}
\setlength{\evensidemargin}{0cm}
\setlength{\topmargin}{-0.5cm}

% ===== ハイパーリンク設定 =====
\hypersetup{
    colorlinks=true,
    linkcolor=blue,
    filecolor=magenta,
    urlcolor=cyan,
    citecolor=blue
}

% ===== 文書情報 =====
\title{【授業名】第X回レポート}
\author{
    学籍番号: 12345678\\
    氏名: 山田 太郎\\
    提出日: \today
}
\date{}

\begin{document}

\maketitle

% ===== 概要 =====
\begin{abstract}
本レポートでは、〇〇について論じる。
まず第1節で背景と目的を述べ、第2節で〇〇の理論について説明する。
第3節では実際の例を示し、最後に第4節でまとめと今後の課題を述べる。
\end{abstract}

% ===== 本文 =====
\section{はじめに}

本レポートの目的は〇〇を明らかにすることである。
この問題は〇〇において重要な意味を持つ。

\section{理論}

\subsection{基本概念}

ここでは、〇〇の基本的な概念について説明する。

\begin{definition}[〇〇の定義]
$X$を〇〇とするとき、以下を満たすものを△△という:
\begin{equation}
    f(x) = \sum_{n=0}^{\infty} a_n x^n
\end{equation}
\end{definition}

\subsection{主要な定理}

\begin{theorem}[〇〇の定理]
任意の$x \in \mathbb{R}$に対して、以下が成り立つ:
\begin{equation}
    \lim_{n \to \infty} \left(1 + \frac{1}{n}\right)^n = e
\end{equation}
\end{theorem}

\begin{proof}
(証明をここに書く)
\end{proof}

\section{例題と考察}

\begin{example}
$f(x) = x^2 + 2x + 1$とするとき、$f(x) = 0$の解を求めよ。
\end{example}

\textbf{解答:}
因数分解すると、
\begin{align}
    f(x) &= x^2 + 2x + 1\\
         &= (x + 1)^2
\end{align}
よって、$x = -1$(重解)

\section{プログラムによる実装}

以下にPythonによる実装例を示す:

\begin{lstlisting}[style=python, caption=計算プログラム]
import numpy as np
import matplotlib.pyplot as plt

def calculate(x):
    """計算を行う関数"""
    return x**2 + 2*x + 1

def plot_function(x_range=(-3, 1), num_points=100):
    """関数をプロットする"""
    x = np.linspace(*x_range, num_points)
    y = calculate(x)
    
    plt.figure(figsize=(8, 6))
    plt.plot(x, y, 'b-', linewidth=2, label='f(x) = x² + 2x + 1')
    plt.axhline(y=0, color='k', linestyle='--', alpha=0.3)
    plt.axvline(x=0, color='k', linestyle='--', alpha=0.3)
    plt.grid(True, alpha=0.3)
    plt.xlabel('x')
    plt.ylabel('f(x)')
    plt.title('二次関数のグラフ')
    plt.legend()
    plt.show()

# メイン処理
if __name__ == "__main__":
    # 関数値の計算例
    test_values = [-2, -1, 0, 1]
    print("関数値の計算結果:")
    for x in test_values:
        y = calculate(x)
        print(f"f({x}) = {y}")
    
    # グラフの描画
    plot_function()
\end{lstlisting}

\section{まとめ}

本レポートでは〇〇について検討した。
その結果、以下のことが明らかになった:

\begin{enumerate}
    \item 〇〇は△△である
    \item □□の条件下では〇〇が成立する
    \item 今後の課題として〇〇が挙げられる
\end{enumerate}

% ===== 付録(必要に応じて) =====
\section*{付録}
\addcontentsline{toc}{section}{付録}

\subsection*{A. 追加の計算結果}
詳細な数値計算結果や補足データをここに記載する。

\subsection*{B. プログラムの実行結果}
\begin{Verbatim}[frame=single, fontsize=\small]
関数値の計算結果:
f(-2) = 1
f(-1) = 0
f(0) = 1
f(1) = 4
\end{Verbatim}

% ===== 参考文献 =====
\begin{thebibliography}{9}
\bibitem{ref1} 著者名, 『書籍タイトル』, 出版社, 2024年.
\bibitem{ref2} Author, A., ``Paper Title,'' \textit{Journal Name}, vol.1, pp.1--10, 2024.
\bibitem{ref3} Smith, J., ``Mathematical Analysis Methods,'' Academic Press, 2023.
\end{thebibliography}

\end{document}