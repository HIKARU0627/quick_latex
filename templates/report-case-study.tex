\documentclass[12pt,a4paper]{ltjsarticle}

% ===== 日本語フォント設定(LuaLaTeX用) =====
\usepackage{luatexja-fontspec}

% ===== 基本パッケージ =====
\usepackage{amsmath,amssymb,amsthm}
\usepackage{graphicx}
\usepackage{xcolor}
\usepackage{hyperref}
\usepackage{booktabs}
\usepackage{float}

% ===== 色の定義 =====
\definecolor{primaryblue}{RGB}{25, 102, 179}
\definecolor{secondaryblue}{RGB}{52, 152, 219}
\definecolor{accentorange}{RGB}{243, 156, 18}
\definecolor{accentgreen}{RGB}{39, 174, 96}
\definecolor{darkgray}{RGB}{52, 58, 64}
\definecolor{lightgray}{RGB}{248, 249, 250}

% ===== ハイパーリンク設定 =====
\hypersetup{
    colorlinks=true,
    linkcolor=primaryblue,
    filecolor=magenta,
    urlcolor=primaryblue,
    citecolor=primaryblue
}

% ===== カスタムコマンド =====
\newcommand{\highlight}[1]{\textcolor{accentorange}{\textbf{#1}}}
\newcommand{\important}[1]{\textcolor{red}{\textbf{#1}}}
\newcommand{\note}[1]{\textcolor{accentgreen}{\textit{#1}}}

% ===== 追加パッケージ =====
\usepackage{enumitem}
\usepackage{fancyvrb}
\usepackage{tcolorbox}
\tcbuselibrary{skins,breakable}

% ===== 文書情報 =====
\title{【授業名】ケーススタディレポート}
\author{
    学籍番号: 12345678\\
    氏名: 山田 太郎\\
    担当教員: ○○ ○○ 教授\\
    提出日: \today
}
\date{}

% ===== ケーススタディ用環境定義 =====
\newtcolorbox{caseframe}[1][]{
    colback=lightgray!20,
    colframe=primaryblue,
    fonttitle=\bfseries,
    title={ケース概要},
    #1
}

\newtcolorbox{analysisframe}[1][]{
    colback=accentgreen!10,
    colframe=accentgreen,
    fonttitle=\bfseries,
    title={分析ポイント},
    #1
}

\newtcolorbox{proposalframe}[1][]{
    colback=accentorange!10,
    colframe=accentorange,
    fonttitle=\bfseries,
    title={提案・改善案},
    #1
}

\begin{document}

% ===== タイトルページ =====
\maketitle

% ===== 概要 =====
\begin{abstract}
本レポートでは、〇〇企業の△△に関するケーススタディを行う。
当該企業が直面した課題を分析し、解決策を検討する。
理論的フレームワークを用いて現状を分析し、
実践的な改善提案を行うことを目的とする。
\end{abstract}

% ===== 目次 =====
\tableofcontents
\newpage

% ===== 本文 =====
\section{はじめに}

\subsection{ケーススタディの目的}
本ケーススタディの目的は以下の通りである:
\begin{enumerate}
    \item 実際の企業事例を通じて理論の実践的応用を学ぶ
    \item 問題分析能力と解決策立案能力を向上させる
    \item 複合的な視点から企業経営を理解する
\end{enumerate}

\subsection{分析手法}
今回のケーススタディでは以下の分析手法を用いる:
\begin{itemize}
    \item SWOT分析(強み・弱み・機会・脅威)
    \item 5フォース分析(競争環境分析)
    \item バリューチェーン分析
    \item 財務分析
\end{itemize}

\section{ケース概要}

\begin{caseframe}
\textbf{企業名:} ○○株式会社 \\
\textbf{業界:} △△業界 \\
\textbf{設立:} 19XX年 \\
\textbf{従業員数:} 約X,XXX名 \\
\textbf{売上高:} XXX億円(20XX年度) \\

\vspace{0.5cm}
\textbf{ケースの背景:}
○○社は長年にわたり△△業界のリーディングカンパニーとして成長してきたが、
近年の市場環境の変化により、従来のビジネスモデルの見直しを迫られている。
特に、デジタル化の進展や新規参入企業の増加により、
競争優位性の維持が困難になっている。
\end{caseframe}

\subsection{事業内容}
○○社の主要事業は以下の通りである:

\begin{enumerate}[label=(\arabic*)]
    \item \textbf{コア事業:} △△の製造・販売
    \item \textbf{関連事業:} □□サービスの提供
    \item \textbf{新規事業:} ××分野への進出
\end{enumerate}

\subsection{財務状況}
過去5年間の主要財務指標を表\ref{tab:financial}に示す。

\begin{table}[H]
\centering
\caption{主要財務指標の推移}
\label{tab:financial}
\begin{tabular}{lrrrrr}
\toprule
指標 & 2019年 & 2020年 & 2021年 & 2022年 & 2023年 \\
\midrule
売上高(億円) & 1,200 & 1,150 & 1,100 & 1,080 & 1,050 \\
営業利益(億円) & 120 & 100 & 85 & 70 & 60 \\
営業利益率(\%) & 10.0 & 8.7 & 7.7 & 6.5 & 5.7 \\
ROE(\%) & 8.5 & 7.2 & 6.1 & 5.0 & 4.2 \\
\bottomrule
\end{tabular}
\end{table}

\section{問題分析}

\subsection{SWOT分析}

\begin{analysisframe}
\textbf{強み(Strengths)}
\begin{itemize}
    \item 長年培った技術力とブランド力
    \item 安定した顧客基盤
    \item 充実した販売ネットワーク
    \item 豊富な財務基盤
\end{itemize}

\textbf{弱み(Weaknesses)}
\begin{itemize}
    \item デジタル技術への対応の遅れ
    \item 組織の硬直化
    \item 新規事業開発能力の不足
    \item 若手人材の確保困難
\end{itemize}

\textbf{機会(Opportunities)}
\begin{itemize}
    \item 新興国市場の成長
    \item デジタル化による新サービス創出
    \item 環境配慮型製品への需要増加
    \item 業界再編による市場シェア拡大
\end{itemize}

\textbf{脅威(Threats)}
\begin{itemize}
    \item 新規参入企業の台頭
    \item 技術革新による製品の陳腐化
    \item 原材料価格の上昇
    \item 規制強化のリスク
\end{itemize}
\end{analysisframe}

\subsection{競争環境分析(5フォース)}

\subsubsection{買い手の交渉力}
顧客の集約度が高く、価格交渉力が強い。特に大口顧客への依存度が高いため、
価格競争に巻き込まれやすい状況にある。

\subsubsection{売り手の交渉力}
主要原材料の供給業者が限定的であり、原材料価格の上昇圧力を受けやすい。

\subsubsection{新規参入の脅威}
参入障壁は比較的高いものの、技術革新により新たなプレイヤーの参入が
現実的となっている。

\subsubsection{代替品の脅威}
デジタル技術の進展により、従来製品を代替する新しいソリューションが
登場している。

\subsubsection{既存競合他社}
業界内の競争は激化しており、価格競争とイノベーション競争の両面で
圧力が増している。

\section{課題の特定}

前述の分析結果から、○○社が直面する主要課題を以下のように特定した:

\begin{enumerate}
    \item \highlight{デジタル変革の遅れ}
    \begin{itemize}
        \item DXへの取り組みが競合他社に比べて遅れている
        \item 社内のデジタルリテラシー向上が急務
    \end{itemize}
    
    \item \highlight{収益性の悪化}
    \begin{itemize}
        \item 営業利益率が年々低下している
        \item コスト構造の見直しが必要
    \end{itemize}
    
    \item \highlight{イノベーション創出力の不足}
    \begin{itemize}
        \item 新商品開発のスピードが遅い
        \item 社内の創造性を促進する仕組みが不足
    \end{itemize}
    
    \item \highlight{人材の確保・育成}
    \begin{itemize}
        \item 優秀な若手人材の獲得が困難
        \item 既存社員のスキルアップが課題
    \end{itemize}
\end{enumerate}

\section{解決策の提案}

\begin{proposalframe}
\subsection{短期的施策(1-2年)}

\textbf{1. デジタル化の推進}
\begin{itemize}
    \item 社内業務プロセスのデジタル化
    \item 従業員向けデジタルスキル研修の実施
    \item 外部DXコンサルタントとの連携
\end{itemize}

\textbf{2. コスト構造の最適化}
\begin{itemize}
    \item 非効率な業務プロセスの見直し
    \item 調達コストの削減
    \item 間接費の圧縮
\end{itemize}

\subsection{中期的施策(3-5年)}

\textbf{1. 新規事業の創出}
\begin{itemize}
    \item 社内ベンチャー制度の導入
    \item オープンイノベーションの推進
    \item 戦略的パートナーシップの構築
\end{itemize}

\textbf{2. 人材戦略の強化}
\begin{itemize}
    \item 多様な人材の積極採用
    \item 人材育成プログラムの充実
    \item 働き方改革の推進
\end{itemize}

\subsection{長期的施策(5-10年)}

\textbf{1. ビジネスモデルの変革}
\begin{itemize}
    \item サービス事業への展開
    \item プラットフォーム型ビジネスの構築
    \item 持続可能な経営モデルの確立
\end{itemize}
\end{proposalframe}

\section{実行計画}

\subsection{優先順位と実行スケジュール}

提案した施策の優先順位と実行スケジュールを表\ref{tab:schedule}に示す。

\begin{table}[H]
\centering
\caption{施策の優先順位と実行スケジュール}
\label{tab:schedule}
\begin{tabular}{lcccp{5cm}}
\toprule
施策 & 優先度 & 開始時期 & 期間 & 期待効果 \\
\midrule
業務プロセスDX & 高 & 即時 & 12ヶ月 & 効率化・コスト削減 \\
デジタル人材育成 & 高 & 3ヶ月後 & 継続 & 組織能力向上 \\
コスト構造見直し & 高 & 即時 & 6ヶ月 & 収益性改善 \\
新規事業開発 & 中 & 6ヶ月後 & 24ヶ月 & 成長基盤構築 \\
働き方改革 & 中 & 6ヶ月後 & 18ヶ月 & 人材確保・定着 \\
\bottomrule
\end{tabular}
\end{table}

\subsection{成功要因}

提案施策を成功させるための重要な要因:

\begin{enumerate}
    \item \textbf{経営トップのコミットメント}
    \begin{itemize}
        \item 変革に対する強いリーダーシップ
        \item 必要な投資の意思決定
    \end{itemize}
    
    \item \textbf{組織全体の意識改革}
    \begin{itemize}
        \item 変化への抵抗を最小化
        \item 新しい文化の醸成
    \end{itemize}
    
    \item \textbf{適切なKPIの設定と監視}
    \begin{itemize}
        \item 定量的な成果指標の設定
        \item 定期的な進捗レビュー
    \end{itemize}
\end{enumerate}

\section{リスクと対策}

\subsection{想定されるリスク}

\begin{enumerate}
    \item \textbf{実行リスク}
    \begin{itemize}
        \item 社内抵抗による実行の遅れ
        \item 必要なスキル・ノウハウの不足
    \end{itemize}
    
    \item \textbf{市場リスク}
    \begin{itemize}
        \item 市場環境の急速な変化
        \item 競合他社の対抗策
    \end{itemize}
    
    \item \textbf{財務リスク}
    \begin{itemize}
        \item 想定以上の投資コスト
        \item 効果の発現遅れ
    \end{itemize}
\end{enumerate}

\subsection{リスク対策}

各リスクに対する具体的な対策を以下に示す:

\begin{itemize}
    \item \textbf{段階的実行:} 小規模パイロットから開始
    \item \textbf{外部支援:} 専門コンサルタントの活用
    \item \textbf{柔軟な計画:} 環境変化に応じた計画修正
    \item \textbf{財務管理:} 厳格な投資回収管理
\end{itemize}

\section{結論}

\subsection{まとめ}

○○社は伝統的な強みを持つ一方で、デジタル化の遅れや競争激化により
厳しい経営環境に直面している。本ケーススタディで提案した施策は、
同社の課題解決と持続的成長を実現するための包括的なアプローチである。

特に重要なのは、\highlight{デジタル変革}と\highlight{人材戦略}を両輪として
推進することである。短期的な収益改善と中長期的な成長基盤の構築を
バランスよく進めることが成功の鍵となる。

\subsection{学んだこと}

本ケーススタディを通じて学んだ主な点:

\begin{enumerate}
    \item 理論的フレームワークの実践的応用の重要性
    \item 複眼的視点による問題分析の必要性
    \item 実行可能性を考慮した提案の価値
    \item 変革におけるリーダーシップの重要性
\end{enumerate}

\subsection{今後の課題}

\begin{itemize}
    \item より詳細な財務シミュレーションの実施
    \item 業界他社との比較分析の深化
    \item ステークホルダー分析の充実
    \item 実装戦略の具体化
\end{itemize}

% ===== 参考文献 =====
\begin{thebibliography}{9}
\bibitem{ref1} Porter, M.E., 『競争戦略論』, ダイヤモンド社, 2018年.
\bibitem{ref2} Kotter, J.P., 『企業変革力』, 日経BP社, 2019年.
\bibitem{ref3} 野中郁次郎, 『知識創造企業』, 東洋経済新報社, 2020年.
\bibitem{ref4} ○○業界動向調査, 経済産業省, 2023年.
\bibitem{ref5} デジタル変革に関する企業調査, 日本生産性本部, 2023年.
\end{thebibliography}

\end{document}