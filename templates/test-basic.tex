\documentclass[12pt,a4paper]{ltjsarticle}

% ===== 日本語フォント設定(LuaLaTeX用) =====
\usepackage{luatexja-fontspec}

% ===== 基本パッケージ =====
\usepackage{amsmath,amssymb,amsthm}
\usepackage{graphicx}
\usepackage{hyperref}
\usepackage{listings}
\usepackage{xcolor}
\usepackage{enumerate}
\usepackage{fancyvrb}
\usepackage{multicol}

% ===== 定理環境の定義 =====
\theoremstyle{definition}
\newtheorem{definition}{定義}[section]

% ===== 文書情報 =====
\title{【授業名】第X回レポート}
\author{
    学籍番号: 12345678\\
    氏名: 山田 太郎\\
    提出日: \today
}
\date{}

\begin{document}

\maketitle

% ===== 概要 =====
\begin{abstract}
本レポートでは、〇〇について論じる。
まず第1節で背景と目的を述べ、第2節で〇〇の理論について説明する。
第3節では実際の例を示し、最後に第4節でまとめと今後の課題を述べる。
\end{abstract}

% ===== 本文 =====
\section{はじめに}

本レポートの目的は〇〇を明らかにすることである。
この問題は〇〇において重要な意味を持つ。

\section{理論}

\subsection{基本概念}

ここでは、〇〇の基本的な概念について説明する。

\begin{definition}[〇〇の定義]
$X$を〇〇とするとき、以下を満たすものを△△という:
\begin{equation}
    f(x) = \sum_{n=0}^{\infty} a_n x^n
\end{equation}
\end{definition}

\section{まとめ}

本レポートでは〇〇について検討した。
その結果、以下のことが明らかになった:

\begin{enumerate}
    \item 〇〇は△△である
    \item □□の条件下では〇〇が成立する
    \item 今後の課題として〇〇が挙げられる
\end{enumerate}

% ===== 参考文献 =====
\begin{thebibliography}{9}
\bibitem{ref1} 著者名, 『書籍タイトル』, 出版社, 2024年.
\bibitem{ref2} Author, A., ``Paper Title,'' \textit{Journal Name}, vol.1, pp.1--10, 2024.
\end{thebibliography}

\end{document}